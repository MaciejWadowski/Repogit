\documentclass[a4paper,11pt]{article}
\usepackage[polish]{babel}
\usepackage[utf8x]{inputenc}
\usepackage[T1]{fontenc}
\usepackage{times}
\usepackage{graphicx}
\usepackage{anysize}
\usepackage{amsmath}
\usepackage{color}
\usepackage{listings}
\usepackage{tikz}
\lstloadlanguages{Ada,C++}

%\marginsize{left}{right}{top}{bottom}
\marginsize{2.5cm}{2.5cm}{2.5cm}{2.5cm}

\begin{document}


\lstset{inputencoding=utf8x,
        extendedchars=\true,
	literate=%
		{ą}{{\k{a}}}1
             {ę}{{\k{e}}}1
             {ć}{{\'c}}1
}

\begin{lstlisting}[frame=lbtr,framesep=2em]
#include <iostream>
using namespace std;

int main()
{
  int s, n, i;
  cout << "Proszę podać liczbę naturalną n : ";
  cin >> n;
  i = 1;
  s = 1;
  
  while (i < n)
  {
    i++;
    s *= i;
  }
  cout << "Silnia: " << s << endl;
}
\end{lstlisting}
\begin{lstlisting}[frame=lbtr,framesep=2em]
#include <iostream>
using namespace std;

int main()
{
  int s, n, i;  
  cout << "Podaj liczbę naturalną, n = ";
  cin >> n;
  s = 1;  
  
  for(i = 2; i <= n; i++) s *= i;  
  
  cout << "Silnia n: " << s << endl;
}
\end{lstlisting}
\end{document}