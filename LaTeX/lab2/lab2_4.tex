\documentclass[a4paper,11pt,twoside]{article}
\usepackage[T1]{fontenc}
\usepackage{amssymb}
\usepackage[utf8]{inputenc}
\usepackage[polish]{babel}
\usepackage{amsthm}
\usepackage{times}
\usepackage{anysize}

\marginsize{1.5cm}{1.5cm}{1.5cm}{1.5cm}
\sloppy 

\theoremstyle{definition}
\newtheorem{df}{Definicja}
\newtheorem{tw}{Twierdzenie}


\begin{document}

\section*{Algebra Boole'a}

Metoda wnioskowania boolowskiego pochodzi z 1847 roku od Boole'a i była rozwijana przez innych matematyków końca XIX-go wieku. Idee te zostały ponownie odkryte w kontekcie nowych zastosowań w naukach przyrodniczych końca XX-go wieku.

\begin{df}
Niech $X$ będzie dowolnym zbiorem, a $n$ dowolną, ustaloną liczbą naturalną. Dowolne przeksztalcenie $d\colon X^n \to X$ nazywamy {\em $n$-argumentowym działaniem} okreslonym w zbiorze $X$, przy czym {\em dzialaniem zero-argumentowym} nazywamy dowolnie ustalony element zbioru $X$.
\end{df}

\begin{df}
Niech $X$ będzie dowolnym zbiorem, a $n$ dowolną, ustaloną liczbą naturalną. {\em Strukturą algebraiczną} nazywamy strukturą składającą się ze zbioru $X$ wraz z pewną liczbą działań $d_i$ ($i = 1,\dots,n$) okrelonych w tym zbiorze. Strukturą algebraiczną zapisujemy w~postaci układu $(X,d_1,\dots,d_n)$.
\end{df}

\begin{df}
Niech $(B,\vee,\wedge,\neg,0,1)$ będzie strukturą algebraiczną, w której $B$ jest niepustym zbiorem, $\vee$ i~$\wedge$ są działaniami dwuargumentowymi, $\neg$ jest działaniem jednoargumentowym, a 0 i 1 działaniami zero-argumentowymi. Strukturę tą nazywamy {\em algebrą Boole'a}, jeżeli działania $\vee,\wedge,\neg,0,1$ są tak okrelone, że spełniają następujące cztery warunki:
\end{df}

\begin{enumerate}
\item Dzialania $\vee$ i $\wedge$ są łączne i przemienne.
\item Działanie $\vee$ jest rozdzielne względem $\wedge$ i odwrotnie.
\item Dla dowolnego $a \in B$:\\
$a \vee ({\neg}a) = 1,$\\
$a \wedge ({\neg}a) = 0$,\\
$a \vee 0 = a,$\\
$a \wedge 1 = a.$
\item Elementy 0 i 1 są różne.
\end{enumerate}

Elementy zbioru $B$ nazywamy {\em stałymi boolowskimi}, zas każdą zmienną przyjmującą wartosci ze zbioru $B$ nazywamy {\em zmienną boolowską}.\\ \\
{\bf Prawa pochłaniania:} \\
$a \vee a = a,$\\
$a \vee (a \wedge b) = a,$\\
$a \wedge a = a,$\\
$a \wedge(a \vee b) = a,$\\
{\bf Prawa de Morgana:}\\
$\neg(a \vee b) = ({\neg}a) \wedge ({\neg}b),$\\
$\neg(a \wedge b) = ({\neg}a) \vee ({\neg}b).$\\

{\em Dwuwartosciową algebrą Boole'a} ($\mathit{BA}$) nazywamy algebrę Boole'a dla której $B=\{0,1\}$, za działania $\vee,\wedge,\neg$ odpowiadają logicznej alternatywie, koniunkcji i negacji.

Stałe boolowskie 0 i~1 wraz ze wszystkimi zmiennymi boolowskimi algebry $\mathit{BA}$ i ich zaprzeczeniami nazywamy {\em literałami boolowskimi}. 

\begin{df}
Niech $\mathit{BA} = (B,\vee,\wedge,\neg,0,1)$ będzie dwuwartosciową algebrą Boole'a. {\em Zbior wyrżeń (formuł) boolowskich}  algebry $\mathit{BA}$ jest najmniejszym zbiorem spełniającym następujące dwa warunki:
\begin{enumerate}
\item Dowolna stała lub zmienna boolowska algebry $\mathit{BA}$ należy do zbioru formuł boolowskich algebry $\mathit{B}.$
\item Jesli $a$, $b$ są formułami boolowskimi algebry $\mathit{BA}$, to rownież $\neg a$, $a \wedge b$ i~$a \vee b$ są formułami boolowskimi algebry $\mathit{BA}$.
\end{enumerate}
Wartosciowanie $W$ wyrażeń (formuł) boolowskich algebry $\mathit{BA}$ jest funkcją przyporządkowującą każdemu wyrażeniu boolowskiemu liczbę ze zbioru $\{0,1\}$. Dla dowolnego wyrażenia boolowskiego $b$, liczbę $W(b)$ nazywamy wartoscią wyrażenia $b$ i obliczamy ją w zwykły sposób, tzn. poprzez wykonanie wszystkich działań występujących w  wyrażeniu $b$ zgodnie z ich okreleniem oraz w kolejnosci wskazywanej przez nawiasy występujące w wyrażeniu $b$.
\end{df}

\begin{df}
Niech $\mathit{BA} = (B,\vee,\wedge,\neg,0,1)$ będzie dwuwartosciową algebrą Boole'a, a~$n$ dowolną, ustaloną liczbą naturalną. Dowolną funkcją $f\colon B^n \to B$ nazywamy {\em funkcją boolowską $n$ zmiennych}.
\end{df}

Funkcję boolowską okrelamy za pomocą odpowiedniego wyrażenia boolowskiego. Można także opisywać funkcję boolowską za pomocą tabelki zawierającej wszystkie możliwe argumenty ze zbioru $\{0,1\}^n$ wraz z odpowiadającymi im wartosciami ze zbioru $\{0,1\}$.\\

Przykład: $f(x_1,x_2,x_3) = (x_1 \vee x_2) \wedge (\neg x_2 \vee x_3)$.

\begin{tw}
Funkcją boolowską $n$-zmiennych $f$ moąna przedstawić w dwóch postaciach:
\begin{enumerate}
\item $f(x) = \bigvee(x_1^{a_1} \wedge\dots\wedge x_n^{a_n}),\textnormal{gdzie }a = (a_1,\dots,a_n) \textnormal{ przebiega zbiór } f^{-1}(1) \subseteq \{0,1\}^n,$
\item $f(x) = \bigwedge(x_1^{a_1} \vee\dots\vee x_n^{a_n}),\textnormal{gdzie }a = (a_1,\dots,a_n) \textnormal{ przebiega zbiór} f^{-1}(0) \subseteq \{0,1\}^n,$
\end{enumerate}
\end{tw}

przy czym oznaczenie $x_i^{a_i}$ jest równe $x_i$, jesli $a_i=1$ i~$\neg x_i$, jesli $a_i=0$.

Postać pierwszą nazywamy {\em alternatywną postacią normalną} (disjunctive normal form) i oznaczamy przez $\mathit{DNF}_f$. Postać drugą nazywamy {\em koniunkcyjną postacią normalną} (conjunctive normal form) i oznaczamy przez $\mathit{CNF}_f$.

Z powodów technicznych szczególnie atrakcyjna jest sytuacja, gdy do przedstawienia funkcji boo\-lowskiej wystarczą dwa tzw. poziomy logiczne: poziom koniunkcji (na którym występuje koniunkcja stałych lub zmiennych boolowskich) i poziom alternatywy (gdzie wyrażenia koniunkcyjne z pierwszego poziomu tworzą alternatywę). Taką postać funkcji boolowskiej nazywamy {\em wielomianem boolowskim}.

\begin{df}
Niech $f$ będzie funkcją boolowską $n$-zmiennych.
\begin{enumerate}
\item {\em Jednomianem boolowskim} (monom) nazywamy dowolne wyrażenie boolowskie będące koniunkcją literałów boolowskich. {\em Kosztem obliczeniowym} jednomianu boolowskiego nazywamy liczbą literałów boolowskich tworzących jednomian boolowski.
\item {\em Wielomianem boolowskim} (polynomial) nazywamy dowolne wyrażenie boolowskie będące alternatywą jednomianów boolowskich. {\em Kosztem obliczeniowym} wielomianu boolowskiego nazywamy sumą arytmetyczną kosztów obliczeniowych wszystkich jednomianów boolowskich tworzących wielomian boolowski.
\item Wielomian boolowski $p$ {\em oblicza funkcją boolowską} $f$ wtedy i~tylko wtedy, gdy\\ $\forall x \in f^{-1}(B)\colon p(x) = f(x)$.
\item Wielomian boolowski $p$ nazywamy {\em wielomianem boolowskim o najmniejszym koszcie obliczeniowym} dla funkcji boolowskiej $f$ wtedy i tylko wtedy, gdy $p$ oblicza $f$ i nie istnieje inny wielomian boolowski obliczający $f$ i mający mniejszy koszt obliczeniowy niż $p$.
\end{enumerate}
\end{df}

Proces prowadzący do przedstawienia funkcji boolowskiej wpostaci wielomianu boolowskiego o najmniejszym koszcie obliczeniowym nazywamy {\em minimalizacją funkcji boolowskiej}. Definicją wielomianu boolowskiego obliczającego daną funkcje boolowską spełnia postać DNF tej funkcji, a zatem dla każdej funkcji boolowskiej istnieje chociaż jeden wielomian boolowski obliczający tą funkcję. Zatem istnieje również wielomian o najmniejszym koszcie obliczeniowym. 

\begin{df}
Niech $f$ będzie funkcją boolowską $n$-zmiennych. 
\end{df}

Funkcją boolowska $f_{imp}(x_1,\dots,x_n) = x_{i_1}^{a_1} \wedge \dots \wedge x_{i_k}^{a_k}$, gdzie $\{x_{i_1},\dots,x_{i_k}\} \subseteq \{x_{1},\dots,x_{n}\}$ oraz $a_i \in \{0,1\}$ (dla $i=1,\dots,k$), nazywamy {\em implikantem funkcji boolowskiej} $f$ wtedy i~tylko wtedy, gdy spełniony jest następujacy warunek: $\forall x \in B^n \colon (f_{imp}(x) = 1 \Rightarrow f(x) = 1)$

Zbiór wszystkich implikantów funkcji $f$ oznaczamy przez $I(f)$.

\begin{df}
Niech $f$ będzie funkcją boolowską $n$-zmiennych i~niech implikant $g \in I(f)$.  Implikant $g$ nazywamy {\em implikantem pierwszym}, jeżli jest implikantem minimalnym ze względu na liczbę czynników. Zbiór wszystkich implikantów pierwszych funkcji $f$ oznaczamy przez $\mathit{PI}(f)$
\end{df}

Implikant pierwszy danej funkcji boolowskiej ma taką własnosć, że odrzucenie z~niego dowolnego czynnika powoduje, że powstała  w ten sposób funkcja nie jest już implikantem.

\begin{tw}
Wielomian boolowski o~najmniejszym koszcie obliczeniowym dla funkcji boolowskiej $f$ jest zbudowany tylko z implikantów funkcji $f$.
\end{tw}
\end{document}





