\documentclass[a4paper,11pt, fleqn]{article}
\usepackage{amssymb}
\usepackage[polish]{babel}
\usepackage[cp1250]{inputenc}   
\usepackage[T1]{fontenc}
\usepackage{graphicx}
\usepackage{anysize}
\usepackage{enumerate}
\usepackage{times}
\usepackage{subfig} 
\usepackage{amsmath}
\usepackage{color}

%\marginsize{left}{right}{top}{bottom}
\marginsize{3cm}{3cm}{3cm}{3cm}

\begin{document}
	
\section*{Ca�ka Riemanna}	
Niech dana b�dzie {\color{blue}funkcja ograniczona} $f\!\colon [a,b] \to \mathbb{R}$. \emph{Sum� cz�sciow�} (Riemanna) nazywa si� liczb�
\[ R_{f,P(q_1,...,q_n)} = \sum_{i=1}^n f(q_i) \cdot  \Delta p_i . \]
Funkcj� $f$ nazywa sie \emph{ca�kowaln� w sensie Riemanna} lub kr�tko \emph{R-ca�kowaln�}, jesli dla dowolnego ci�gu normalnego $(P^k)$  podzia��w przedzia�u $[a,b]$ istnieje (niezale�na od wyboru punkt�w porednich) granica
\[R_f = \lim_{k \to \infty} R_{f,P^k(q_1^k,...,q_{n_k}^k) }\]
nazywana wtedy \textbf{ca�k� Riemanna} tej funkcji. R�wnowa�nie: je�eli istnieje taka liczba $R_f$
�e dla dowolnej liczby rzeczywistej $\epsilon > 0$ istnieje taka liczba rzeczywista $\delta > 0,$  �e dla dowolnego podzia�u $P(q_1,...,q_n)$ o srednicy $ P(q_1,...,q_n) < \delta;$ b�d� te� w j�zyku rozdrobnie�: �e dla dowolnej liczby rzeczywistej $\delta > 0$ istnieje taki podzia� $ S(t_1,...,t_m) $ przedzial�u $[a,b],$ �e dla ka�dego podzia�u $P(q_1,...q_n)$ rozdrabniaj�cego $S(t_1,...,t_m)$, zachodzi
\[|R_{f,P(q_1,...,q_n)} - R_f | < \epsilon \]
Funkcj� $f$ nazywa si� wtedy \emph{ca�kowaln� w sensie Riemanna (R-ca�kowaln�)}, a liczb� $R_f$ jej \textbf{ca�k� Riemanna}. 

\end{document}