\documentclass[a4paper,12pt]{article}
\usepackage[T1]{fontenc}
\usepackage[utf8]{inputenc}
\usepackage{amssymb}
\usepackage[polish]{babel}
\usepackage{amsthm}
\usepackage{times}
\usepackage{anysize}
\usepackage{array}

\marginsize{1.5cm}{1.5cm}{1.5cm}{1.5cm}
\sloppy 

\theoremstyle{definition}
\newtheorem{df}{Definicja}


\begin{document}

% \maketitle

\section{Zbiory rozmyte}

Pojęcie zbioru rozmytego zostało wprowadzone przez L. A. Zadeha w 1965. Stosowanie zbiorów rozmytych w systemach sterownia pozwala na dokładniejsze odwzorowanie pojęć stosowanych przez ludzi, które często są subiektywne i nieprecyzyjne. Stopniowe przejęcie między przynależnocią do zbioru a jej brakiem pozwala nam uniknąć scisłej klasyfikacji elementów, która często jest niemożliwa. Logika rozmyta jest uogólnieniem logiki klasycznej. 

Koncepcja zbiorów rozmytych wyrosła na gruncie logicznego formalizowania pomysłu, aby elementom zbioru przypisywać tzw. stopień przynależnosci do zbioru, okreslający wartosć prawdziwosci danego wyrażenia za pomocą liczb z przedziału $[0,1]$. 

\begin{df}
\textbf{Zbiorem rozmytym} $A$ w pewnej niepustej przestrzeni $X$ nazywamy zbiór uporządkowanych par:

\begin{equation} \label{eq1}
 A  = \{(x, \mu_A(x))\!:x \in X \},
\end{equation}
gdzie:

\begin{equation} \label{eq2}
\mu_A\!:\!X\rightarrow[0, 1] 
\end{equation}

jest \textbf{funkcją przynależosci} zbioru $A$.
Wartosc funkcji $\mu_A(x)$ w punkcie $x$ \textbf{stopniem przynaleznosci} $x$ do zbioru $A$.
\end{df}


Zbiory rozmyte $A$, $B$, $C$ itd. względem $X$ nazywamy również \textbf{podzbiorami rozmytymi} w $X$.
Zbiór wszystkich zbiorów rozmytych w $X$ oznaczamy przez $F(X)$. Zbiory klasyczne można interpretować jako zbiory rozmyte z funkcją przynależnosci przyjmującą tylko wartosci 0 i 1.

\medskip
\noindent
{\bf Przykład:} Niech $A$ oznacza zbiór niskich temperatur i niech $X = [-5,50]$. Funkcja przynależnosci może być okreslona następująco:
\begin{equation} \label{eq3}
\mu_A = \left\lbrace
\begin{array}{ll}
1  &:x < 10, \\
-\frac{1}{10} + 2 & :10 \leqslant x \leqslant 20, \\
0 &:x > 20.
\end{array}\right.
\end{equation}
W przypadku, gdy ustalony zbiór $X$ jest skończony, tzn. $X = \{x_1,x_2,...,x_n\}$, funkcje przynależnosci zbiorów rozmytych można przedstawiać za pomocą tabelek, stosować zapis w postaci sumy:
\begin{equation} \label{eq4}
A = \sum_{i=1}^n\frac{\mu_A(x)}{x},
\end{equation}
(kreski ułamkowe i znaki sumy należy rozumieć czysto symbolicznie) lub podawać elementy zbioru w postaci par:
\begin{equation} \label{eq5}
 A = \{x_1,\mu_A(x_2)),(x_2,\mu_A(x_1)),...,(x_n,\mu_A(x_n))\}.
\end{equation}

\end{document}

